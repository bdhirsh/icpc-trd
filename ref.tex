\documentclass[a4paper,oneside,twocolumn]{article}
\usepackage[utf8]{inputenc}
\usepackage[english]{babel}

\usepackage[T1]{fontenc}
\usepackage[sc]{mathpazo}
\linespread{1.185}

%\usepackage{eulervm}

\usepackage[landscape,left=.75cm,right=.5cm,top=2.1cm,bottom=.5cm]{geometry}

\setlength{\columnseprule}{0.25pt}
\usepackage{fancyhdr}
\pagestyle{fancy}
\fancyhf{}
\fancyhead[R,RO]{\bfseries\thepage}
\fancyhead[L,LO]{University of Pennsylvania}

\usepackage{listings}

\usepackage{sectsty}
\sectionfont{\normalsize}

\usepackage{xcolor}
\usepackage{textcomp}

\usepackage{fancyvrb}
\fvset{tabsize=2}

\lstset{language=C++,
                basicstyle=\ttfamily,
                keywordstyle=\color[rgb]{0,0,0.7},
                stringstyle=\color[rgb]{0.5,0,0},
                commentstyle=\color[rgb]{0,0.5,0},
                morecomment=[l][\color[rgb]{0.5,0.5,0}]{\#},
                breaklines=true,
                tabsize=2
}

\newcommand{\SECTION}[2]{\section*{#1} \addcontentsline{toc}{subsection}{#1} #2 \begin{center}\rule{400pt}{0.25pt}\end{center}}
\newcommand{\sourcefile}[1]{\lstinputlisting{src/#1}}
\newcommand{\Csourcefile}[1]{\lstinputlisting[language=C++]{src/#1.cpp}}
\newcommand{\Cnotppsourcefile}[1]{\lstinputlisting[language=C++]{src/#1.c}}
\newcommand{\Psourcefile}[1]{\lstinputlisting[language=Python]{src/#1.py}}
%\begin{center}\textbf{#1.cpp}\end{center}

\usepackage{amsmath}
\usepackage{amsfonts}
\usepackage{amssymb}
\usepackage{amsthm}
\newcommand{\NN}{\mathbb{N}}
\newcommand{\ZZ}{\mathbb{Z}}
\newcommand{\QQ}{\mathbb{Q}}
\newcommand{\RR}{\mathbb{R}}
\newcommand{\then}{\Longrightarrow}
\newcommand{\floor}[1]{\left\lfloor#1\right\rfloor}
\newcommand{\ceil}[1]{\left\lceil#1\right\rceil}
\newcommand{\paren}[1]{\left(#1\right)}
\newcommand{\brackets}[1]{\left[#1\right]}
\newcommand{\braces}[1]{\left\{#1\right\}}
\newcommand{\abs}[1]{\left\lvert#1\right\rvert}
\newcommand{\nequiv}{\not\equiv}
\newcommand{\ds}{\displaystyle}
\newcommand{\bigO}{\mathcal{O}}
\newcommand{\norm}[1]{\abs{\abs{#1}}}
\newcommand{\degree}{\ensuremath{^\circ}}
\newcommand{\defun}[5] {
    \begin{array}{rrcl}
#1: & #2 & \longrightarrow & #3 \\
    & #4 & \longmapsto & #5
    \end{array}
}
\renewcommand{\le}{\leqslant}
\renewcommand{\ge}{\geqslant}
\renewcommand{\epsilon}{\varepsilon}

\newcommand{\stirfst}[2]{\genfrac{[}{]}{0pt}{}{#1}{#2}}
\newcommand{\stirsnd}[2]{\genfrac{\{}{\}}{0pt}{}{#1}{#2}}
\newcommand{\bell}[1]{\mathcal B_{#1}}
\newcommand{\seg}[1]{\overline{#1}}

\newcommand{\bmat}[1]{\begin{bmatrix}#1\end{bmatrix}}
\newcommand{\vmat}[1]{\begin{vmatrix}#1\end{vmatrix}}





\title{ACM ICPC Reference}
\author{University of Pennsylvania}

\begin{document}
\footnotesize{}
\maketitle
\thispagestyle{fancy}

\section{Data Structures}

\SECTION{Treap (balanced binary search tree)} {
  \Csourcefile{\detokenize{treap}}
}

\section{Geometry}

\SECTION{Welzl's algorithm (minimum enclosing circle} {
  \Csourcefile{\detokenize{welzl}}
}

\SECTION{Monotone chain (convex hull) and rotating calipers (farthest pair)} {
  \Csourcefile{\detokenize{convexhull-rotatingcaliper}}
}

\SECTION{3d Convex Hull} {
  \Csourcefile{\detokenize{3d-convex-hull}}
}

\section{Graph}

\SECTION{Min cost flow} {
  \Csourcefile{\detokenize{mincostflow}}
}

\SECTION{Dinic's ($VE^2$ max flow)} {
  \Csourcefile{\detokenize{dinic}}
}

\SECTION{Edmond's algorithm (unweighted general matching)} {
  \Csourcefile{\detokenize{edmonds-blossom}}
}

\SECTION{Link cut tree} {
  \Csourcefile{\detokenize{link-cut}}
}

\SECTION{Hungarian algorithm} {
  \Cnotppsourcefile{\detokenize{hungarian/hungarian}}
}


\section{Strings}

\SECTION{KMP (linear string search)} {
  \Csourcefile{\detokenize{kmp}}
}

\SECTION{Manacher (max palindrome substring)} {
  \Csourcefile{\detokenize{manacher}}
}

\SECTION{Suffix array} {
  \Csourcefile{\detokenize{suffix-array}}
}

\section{Math}

\SECTION{Chinese Remainder Theorem} {
  \Csourcefile{\detokenize{crt}}
}

\SECTION{Fast Fourier Transform} {
  \Csourcefile{\detokenize{fft}}
}

\SECTION{Simplex Algorithm (Linear programming)} {
  \Csourcefile{\detokenize{simplex}}
}

\SECTION{Gaussian elimination} {
  \Csourcefile{\detokenize{gaussian-elim}}
}

\SECTION{Karatsuba multiplication} {
  \Csourcefile{\detokenize{karatsuba}}
}

\SECTION{Miller-Rabin (probabilistic primality testing)} {
  \Csourcefile{\detokenize{miller-rabin}}
}


\end{document}
